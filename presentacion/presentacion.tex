\documentclass[a4paper,12pt]{article}

\usepackage[left=2.5cm, right=2.5cm, top=3cm, bottom=3cm]{geometry}
\usepackage{amsmath, amsthm, amssymb}
\usepackage[spanish]{babel}
\usepackage{cite}
\usepackage{graphicx}
\usepackage{float}
\usepackage{url}

\bibliographystyle{plain} 

\begin{document}
\title{Presentacion}
\author{Dario Lopez Falcon}
\date{Julio, 2023}
\maketitle

\begin{abstract}
   El objetivo de esta presentacion es esplicar para que sirve Moogle! , ademas de dar una 
   pequeña guia al usuario para que sepa como realizar las busquedas y que cosas puede o no hacer.  
\end{abstract}

\section{MOOGLE! :}\label{sec:intro}

Moogle! es un programa que dada una consulta introducida por el usuario , tiene como objetivo realizar 
una busqueda rapida en una base de datos (de tipo .txt), dando como resultado una lista de los documentos 
de dicha base de datos ordenada de  mayor a menor segun que tan relevante es dicho documento respecto a la 
consulta realizada

Moogle! funciona con documentos que esten en español o ingles o cualquier idioma que comparta el abecedario 
con estos dos.

\section{Pasos a seguir para realizar una consulta}\label{sec:ps}


\begin{enumerate}
    \item Introduzca la frase o expresion ,etc a la cual quiera hacerle la busqueda en la barra de 
    busqueda. 
    \item Presione el boton de busqueda que aparece a la derecha de la barra o simplemente presione enter.
    \item Recibira la lista que mencionamos al inicio , pero ademas de esto , cada tiyulo vendra acompañado 
    de un pequeño fragmento del documento que guarda relacion con la consulta realizada, puede aparecer la 
    query en su totalidad o solo parte de esta.
    
    \begin{enumerate}
        \item en caso de que no se encuentren documentos relacionados con la busqueda aparecera en pantalla 
        un cartel diciendo lo mismo.
        \item Si no introduce nada en la consulta no aparecera ningun documento , solo se mostrara un cartel 
        pidiendole que introduzca una nueva busqueda.
    \end{enumerate}
    \item utilizando los caracteres ! , \^ , * se pueden realizar busquedas mas personalizadas 
    \begin{itemize}
        \item al utilizar el caracte ! delante de una palabra (tiene q estar pegado a la palabra)
 se le mostraran como resultado todos los txt que tengan relevancia con la query , pero solo mostrara 
 los que no contienen a la palabra que se le puso el ! delante (ojo: si solo escribe una palabra acompa;ada del ! no le saldra ningun resultado)
 De igual forma funciona el caracter \^ , solo que te mostrara solo los txt q si contienen a la palabra que acompaña
        \item El caracter * se utiliza de igual forma que los otros, per lo que hace es darle el doble de relevancia a la 
 palabra que acompaña, por lo que la relevancia de la palabra seria 2 elevado a la cantidad de * que tenga delante
 porque se pueden poner mas de un * delate , lo que todos tienen q estar juntos,
        \item Pruebe hacer busqueda con algunas combinaciones de estos caracteres lo que siempre ponga los caracteres 
        delante de la palabra y pegada a ella y no poner mas de un ! o un \^ (el programa no explota al hacerlo pero 
        no tiene ningun sentido) y por supuesto no poner juntos !\^ porque es algo absurdo,una palabra no poede estar y no estar a la vez XD.  
    \end{itemize}
\end{enumerate}
*Un detalle importante es que por la forma en esta diseñado Moogle! la primera busqueda es la que mas va a tardar 
pero despues todas las demas consultas son en tiempo real.(espero en el futuro poder cambiar eso para que el programa 
precalcule todo antes de realizar las busquedas para que estas sean en tiempo real)

Espero que el diseño sea de su agrado , se que es sencillo pero queria hacerlo un poco diferente(Tambien espero poder hacer mejoras en el futuro)



\end{document}
