\documentclass[a4paper,12pt]{article}
% En dependencia de como instalaron latex (básico, extra, full), puede que para
% usar ciertos paquetes sea necesario instalarlos adicionalmente. Para ellos
% buscan en internet qué deben instalar (de acuerdo a su sistema operativo)
% para usar cada paquete en específico.

% Modificar las propiedades de la hoja.
\usepackage[left=2.5cm, right=2.5cm, top=3cm, bottom=3cm]{geometry}

% Paquetes para el rednerizado de estructuras matemáticas, teoremas, símbolos, etc.
% AMS es una familia de paqueted
\usepackage{amsmath, amsthm, amssymb}

% Usar plantilla en español. Puede ser que tengan que instalar
% el paquete del lenguaje español. Para ellos busquen en internet
% cómo instalarlo de acuerdo a su sistema operativo.
% 
% En linux:
%   - Ubuntu: apt install texlive-lang-spanish
%   - Arch: pacman -S texlive-langspanish
\usepackage[spanish]{babel}

% Para agregar citas bibliográficas
\usepackage{cite}

% Para insertar imágenes
\usepackage{graphicx}

% Para agregar links
\usepackage{url}

% Estilo de la bibliografía
\bibliographystyle{plain} % Otros estilos: apalike, alpha, abbrv, etc

\begin{document}
\title{Documento de ejemplo para el uso de \LaTeX}
\author{Jorge Morgado Vega}
\date{Julio, 2023}
\maketitle

% La clase 'article' permite declarar un abstract (resumen)
\begin{abstract}
    El abstract de un artículo es un párrafo de aproximadamente 250 palabras que
    resume la escencia del artículo. El mismo puede contener una pequeña
    introducción al tema, lo que se realizó en el trabajo y los resultados
    principales obtenidos.
\end{abstract}

\section{Introducción}\label{sec:intro}

La introducción debe escribirse desde lo más general a lo más específico.  Se
comienza hablando del tema o el campo donde se aplica el estudio realizado, así
como la importancia de este tipo de estudios. Se deben citar además estudios
realizados anteriormente \cite{hu2016entangled} que tienen relación con su
trabajo. Luego se explica el motivo de realización del trabajo.

Al final de la introducción se puede escribir de forma opcional un párrafo que
explique como está estructurado el documento. Por ejemplo;

La Sección \ref{sec:ent} presenta varios de los entornos más comunmente usados.
En la Sección \ref{sec:math} se explican las diferentes formas de escribir
expresiones matemáticas. La Sección \ref{sec:img-and-tables} muestra cómo se
pueden insertar Imágenes y tablas al documento.

\section{Entornos comunes}\label{sec:ent}

\subsection{Entorno \texttt{center}}\label{sub:center}

Se utiliza para centrar el texto:

\begin{center}
    Texto centrado
\end{center}

\subsection{Entorno \texttt{itemize}}\label{sub:itemize}

Se usa para nombrar elementos que no tienen un orden significativo:

\begin{itemize}
    \item Un elemento
    \item Otro elemento
    \begin{itemize}
        \item Un sub elemento
        \item Otro sub elemento
    \end{itemize}
    \item[-] Un elemento con otro símbolo
    \item Otro elemento con el símbolo original
\end{itemize}

\subsection{Entorno \texttt{enumerate}}\label{sub:enumerate}

Se usa para nombrar elementos que tienen un orden:

\begin{enumerate}
    \item Primer elemento
    \item segundo elemento
    \begin{enumerate}
        \item Un sub elemento
        \item Otro sub elemento
    \end{enumerate}
    \item Último elemento
\end{enumerate}

\section{Expresiones matemáticas}\label{sec:math}

Existen diversos comandos y entornos que permiten escribir todo tipo de
expresiones matemáticas.

El más simple es mediante el uso del caracter \$. Este se emplea cuando se
quiere insertar una expresión dentro de un texto. Por ejemplo, si $A \subseteq
B$ y $B \subseteq C$ entonces se puede decir que $A \subseteq C$.

Si se usa el caracter \$ doble entonces al renderiza la expresión se har un
salto de línea en ambos extremos y se mostrará centrada. Por ejemplo, si $$x \ge
y \wedge y \ge z$$ entonces $x \ge z$.

\subsection{Ecuaciones}\label{sub:eq}

Las ecuaciones tmb pueden ser enumeradas y referenciadas mediante el uso del
entorno \texttt{equation}:

\begin{equation}\label{eq:grav}
    F = G\frac{m_1m_2}{r^2}
\end{equation}

La fórmula \ref{eq:grav} ilustra la fuerza de gravedad que existe entre un
cuerpo de masa $m_1$ y un cuerpo de masa $m_2$ que se encuentran separados por
una distancia $r$.

Para una guía de todo lo que se puede realizar con respecto a estructuras
matemáticas pueden referirse al manual de \texttt{amsmath}\footnote{Disponible
en: \url{http://www.ams.org/arc/tex/amsmath/amsldoc.pdf}}

\section{Imágenes}\label{sec:img-and-tables}

Como se mostró en la presentación, se pueden usar el entorno \texttt{figure}
junto con el comando \texttt{includegraphics} para agregar imágenes al
documento.  La opción \texttt{[h]} del entorno \texttt{figure} indica a latex que
inserte la figura en la misma posición en la que se declaró en el documento
\texttt{.tex}.

\section{Conclusiones}\label{sec:concl}

A medida que empiecen a usar \LaTeX{} para escribir sus documentos van a empezar a
tener muchas dudas, en esos casos internet será su mejor amigo. Pueden contar
además con la ayuda de sus profesores.

% Empezar una nueva página.
\newpage

% El índice se puede generar de forma automática. El siguiente comando analiza
% la estructura del documento (secciones, subsecciones, etc) y genera la tabla
% de contenidos. (No tiene que ir al final del documento obligatoriamente, puede
% estar al inicio también)
\tableofcontents

\newpage
\bibliography{bibliography}
\end{document}